\documentclass{article}
\usepackage[utf8]{inputenc}

\usepackage[margin=1in]{geometry}

\setlength{\parindent}{0pt}
\setlength{\parskip}{.5em}

\title{CS 5033: Final Project Proposal}
\author{Shane Flandermeyer}
\date{}

\begin{document}

\maketitle

\section{Problem Statement}
The rapid evolution of wireless communications technologies such as 4G/5G and the
internet-of-things (IoT) has radically altered daily life. These technologies
utilize the radio frequency (RF) portion of the electromagnetic spectrum,
which is a finite resource. To effectively access and regulate the spectrum, it
is essential that the next generation of wireless devices be able to rapidly
classify and monitor signals in the spectrum. Information gained from sensing the spectrum can
be used for tasks such as cognitive radio, interference detection, and dynamic
spectrum access \cite{Kulin2018}. Traditionally, signal detection and classification has been
accomplished through static filtering and signal processing using expert features, which for the
most part is not data-adaptive \cite{Ariananda2009}.

In this project, I propose to apply deep learning to the modulation recognition
problem. I will formulate this as a multi-class classification task that I will solve
using convolutional neural networks (CNNs). The network(s) will take RF data as
input and output the signal type along with a confidence score. My central research question is \textit{how
  accurate are convolutional neural networks for signal classification in a
  congested spectrum?}

To address the above question, I propose the following
research activities:

-- create a framework to easily synthesize in-phase/quadrature (IQ) data,
which is a sampled version of the signal received by an RF antenna

-- Develop a CNN that can localize and classify RF signals from raw IQ data

-- Evaluate the network on a testing set which will also be synthetically generated

-- Stretch goal 1: Explore alternative representations of the IQ data, such as
spectrograms (time-frequency plots) and constellation diagram (real-imaginary plots)
For example, rather than training on the raw data itself the model might be trained
on the signal spectrogram (a time-frequency plot). This would allow the
model to better distinguish between signals that are similar in the time domain
but have different frequency content and vice-versa.

-- Stretch goal 2: Collect over-the-air (not simulated) data using a commercial software-defined radio.

-- Stretch goal 3: Extend the data synthesizer to simulate radar signals

\section{Methodology}
The goal of this project is to implement (and hopefully improve upon)
the methodology from \cite{OShea2016a}, which uses raw IQ data to train a
CNN to classify 11 modulation types. Since my primary focus is dataset
generation, this network will be constructed using libraries such as Keras and
Tensorflow. Time permitting, I would also like to explore the approach taken
in \cite{Vagollari2021}, which uses the YOLO algorithm \cite{Redmon2016} to
train a CNN from spectrogram images. The YOLO algorithm is designed for
real-time implementation, which is an obvious requirement for deployment in
spectrum-monitoring cognitive radios.

\section{Data Preparation}
A major component of this project will be to develop an RF data
synthesis tool. Since radio communications signals are already synthetically generated in
practice, it should be possible to
mass-produce synthetic IQ data that is representative of real-world scenarios.
The synthesis architecture will largely follow the general model outlined in
\cite{OShea2016}. I will implement it in Python/GNU Radio, and it will be able
to simulate phenomena such as thermal noise, channel effects (e.g., multipath
and fading), and center frequency/sampling offsets. Coupled with
the methodology outlined above, my project primarily falls under category I (Application).

\section{Evaluation Plan}


\bibliographystyle{plain}
\bibliography{../reference}

\end{document}
